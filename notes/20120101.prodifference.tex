\bibliography*{sample} %% setup reference database.
\bibliographystyle*{bibstyle}     %% setup reference style.

\section{快速多极边界元的四叉树生成算法}

快速多极边界元方法借助了自适应四叉树,给出了边界元的加速算法。在使用这四叉树的边界元算法中,
使得内存的使用量,和迭代计算的代价得到了改善。

下面给出自适应四叉树的生成算法:
\begin{enumerate}
	\item [\textbf{Step 0.}] 给定树节点中最大单元数$emax$。  
	\item [\textbf{Step 1.}] 初始化根结点,并设定其所在的层为第0层,并令$depth = 0$。
	\item [\textbf{Step 2.}] 设$k = depth$, 遍历第$k-1$层的非叶子树节点,\footnote{在这一步之前并不知道$k-1$层的节点是否是叶子节点。}
		\begin{enumerate}
			\item [\textbf{Case 1.}] 如果树节点中的单元数小于等于$emax$, 标记该树节点为叶子节点。
			\item [\textbf{Case 2.}] 如果树节点中的单元数大于$emax$, 标记该树节点为非叶子节点,并调用四分树节点算法,生成下一层的树节点。
		\end{enumerate}
	\item [\textbf{Step 3.}] 若有新的树节点产生,$depth = depth +1$,并转\textbf{Step 2},否则转\textbf{Step 4}。
	\item [\textbf{Step 4.}] 结束算法。
\end{enumerate}

四分树节点的算法:
\begin{enumerate}
	\item 读取设定的树节点最大单元数$emax$;
	\item 计算该节点四个象限中的单元数;
	\item 遍历四个象限,
		\begin{enumerate}
			\item 若该象限中的单元数不为0,则创建树节点,并将该子节点加入到树节点列表,若节点中的单元数大于$emax$,则将树节点标记为非叶子节点,
				否则将该树节点标记为叶子节点。
			\item 若该象限中的单元数为零,转到计算下一个象限。
		\end{enumerate}
	\item 算法结束。
\end{enumerate}


遍历树算法




%\putbib[sample]             %% list reference.

\newpage

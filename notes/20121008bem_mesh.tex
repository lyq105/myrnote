
%% setup reference database.
\bibliography*{20121008bem_mesh}
\bibliographystyle*{bibstyle}     %% setup reference style.
%% Section title
\section{关于边界元网格的考虑}

在边界元计算中,首先要生成边界元网格,这样的网格可以通过有限元网格的生成来实现。
一般来说,只要提取出有限元网格的表面就可以当作边界元计算的网格了。

\subsection{gmsh对边界元网格生成的支持}
下面给出gmsh网格数据文件的例子\cite{geuzaine2009gmsh}.
\begin{lstlisting}
		 $MeshFormat
     2.2 0 8
     $EndMeshFormat
     $Nodes
     6                      six mesh nodes:
     1 0.0 0.0 0.0            node #1: coordinates (0.0, 0.0, 0.0)
     2 1.0 0.0 0.0            node #2: coordinates (1.0, 0.0, 0.0)
     3 1.0 1.0 0.0            etc.
     4 0.0 1.0 0.0
     5 2.0 0.0 0.0
     6 2.0 1.0 0.0
     $EndNodes
     $Elements
     2                      two elements:
     1 3 2 99 2 1 2 3 4       quad #1: type 3, physical 99, elementary 2, nodes 1 2 3 4
     2 3 2 99 2 2 5 6 3       quad #2: type 3, physical 99, elementary 2, nodes 2 5 6 3
     $EndElements
     $NodeData
     1                      one string tag:
     "A scalar view"          the name of the view ("A scalar view")
     1                      one real tag:
     0.0                      the time value (0.0)
     3                      three integer tags:
     0                        the time step (0; time steps always start at 0)
     1                        1-component (scalar) field
     6                        six associated nodal values
     1 0.0                  value associated with node #1 (0.0)
     2 0.1                  value associated with node #2 (0.1)
     3 0.2                  etc.
     4 0.0
     5 0.2
     6 0.4
     $EndNodeData	
\end{lstlisting}
这是一个gmsh2的文件格式,这个文件将所有的单元都写在文件里面了。
\subsection{netgen对边界元网格生成的支持}
下面给出netgen的两种网格文件的例子\cite{schoberl1997netgen}。
\begin{lstlisting}
surfacemesh
8
         0          0          0 
         0          0          1 
         1          0          0 
         0          1          0 
         1          0          1 
         0          1          1 
         1          1          0 
         1          1          1 
12
       1       4       7
       7       3       1
       4       1       2
       2       6       4
       8       7       4
       6       8       4
       7       8       5
       3       7       5
       8       6       2
       2       5       8
       3       5       2
       1       3       2
\end{lstlisting}
这一个文件是表面网格数据,他的包含的是8个点和12个三角形面,覆盖了一个正方体。
\begin{lstlisting}
8
  0.000000  0.000000  0.000000
  0.000000  0.000000  1.000000
  1.000000  0.000000  0.000000
  0.000000  1.000000  0.000000
  1.000000  0.000000  1.000000
  0.000000  1.000000  1.000000
  1.000000  1.000000  0.000000
  1.000000  1.000000  1.000000
6
   1          4        2        6        8
   1          8        7        2        5
   1          3        2        1        7
   1          3        5        2        7
   1          1        2        4        7
   1          8        7        4        2
12
   1            1        4        7
   1            7        3        1
   2            4        1        2
   2            2        6        4
   3            8        7        4
   3            6        8        4
   4            7        8        5
   4            3        7        5
   5            8        6        2
   5            2        5        8
   6            3        5        2
   6            1        3        2
\end{lstlisting}
这一个文件是体网格数据,他的包含的是8个点和12个三角形面以及6个四面体单元,这些单元剖分了一个正方体, 其中单元的数据行开始都是以其标记开始的。

\subsection{总结}
需要注意的是,对于生成的表面网格,都是外法向反时针的顺序。因此就不需要再对得到的外法向做方向调整了。

今后的主要支持方向还是gmsh,因为还有数据文件,数据文件还指望gmsh。
%% list references if any.
\putbib[20121008bem_mesh]
\newpage

\bibliography*{20120327.sample} %% setup reference database.
\bibliographystyle*{bibstyle}     %% setup reference style.

\section{计算边界积分的退化核}

若核函数可以表示为\cite{beatson1997short}
\[
K(x,y) = \sum_{k=1}^{p}\varphi_k(x)\varsigma_k(y)
\]
那么如下的矩

\begin{equation}
	A_k = \sum_{i=1}^{N} \wedge_i\varphi_k(y_i).
\end{equation}
可以先计算好,要计算源点处的函数值则只需要做$p$次乘法和$p-1$次加法。
\begin{equation}
	u(x) = \sum_{i=1}^{p}A_k\varsigma_k(x).
\end{equation}
则要是计算在$N$个点处的位势值的话,只需要$O(N)$的计算量。

特别地在边界元中,若核函数可以展开为远场和近场同时适用的级数形式,那么计算则
十分的简便。

\putbib[20120327.sample]             %% list reference.

\newpage

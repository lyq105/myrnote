\section{libmesh备忘}

使用triangle产生一个具有孔洞的三角网格。
\begin{lstlisting}[language=C]
	Mesh mesh(2);
	mesh.add_point(Point(-1,-1));
	mesh.add_point(Point(1,-1));
	mesh.add_point(Point(1,1));
	mesh.add_point(Point(-1,1));

	TriangleInterface t(mesh);

	// Customize the variables for the triangulation
	t.desired_area()       = .0001;
	t.triangulation_type() = TriangleInterface::PSLG;
	t.smooth_after_generating() = true;
	PolygonHole hole_1(Point(0.,  0.), // center
	0.51,              // radius
	100);               // n. points
	std::vector<Hole*> holes;
	holes.push_back(&hole_1);
	t.attach_hole_list(&holes);
	t.triangulate();
	mesh.prepare_for_use(false);
\end{lstlisting}
注意最后一行的prepare\_for\_use(),必须要调用才能使用网格。他的原型是
\begin{lstlisting}[language=C]
void libMesh::MeshBase::prepare_for_use	(	const bool 	skip_renumber_nodes_and_elements = true	)		
\end{lstlisting}
它包含三个步骤
\begin{itemize}
	\item 1.) call find\_neighbors()
	\item 2.) call partition() 
	\item 3.) call renumber\_nodes\_and\_elements()
\end{itemize}
\begin{lstlisting}[language=C]
	FEType fe_type(FIRST, LAGRANGE); // 指定逼近的单元族
  equation_systems.get_system("Poisson").add_variable("u", fe_type); //将逼近与变量结合起来
\end{lstlisting}
solve 包含两个基本的步骤,一个是调用组装函数,另一个是求解线性方程组。
\begin{lstlisting}[language=C]
 equation_systems.get_system("Poisson").solve();
\end{lstlisting}

这样可以直接输出计算结果到Tecplot
\begin{lstlisting}[language=C]
	TecplotIO(mesh).write_equation_systems ("squre_tri_res.plt",equation_systems);
\end{lstlisting}
定义第一类边界条件,下面的代码表示将3号边界上的u和v均设置为0.
\begin{lstlisting}[language=C]
 std::set<boundary_id_type> boundary_ids;
          boundary_ids.insert(3);
std::vector<unsigned int> variables(2);
          variables[0] = u_var; variables[1] = v_var;				
ZeroFunction<> zf;
DirichletBoundary dirichlet_bc(boundary_ids,
variables,
&zf);
\end{lstlisting}


\lstinputlisting[language=C]{codes/thermal_2D.C}
